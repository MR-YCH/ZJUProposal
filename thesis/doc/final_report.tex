%%%%% --------------------------------------------------------------------------------
%%
%%%%******************************* Final Report *************************************
%%
%%% ++++++++++++++++++++++++++++++++++++++++++++++++++++++++++++++++++++++++++++++++++
\setcounter{section}{0}

\cleardoublepage
\section{\FakeBold{引言}} % (fold)
\label{sec:引言}
	
	本文只规定格式,标题仅做为参考,具体请根据研究内容自行拟定。
	正文使用宋体小四,英文使用Times new roman小四;正文行距为1.5倍行距。
	正文字数不少于1万字;按双面打印排版:各一级标题及参考文献从奇数页开始,页码奇偶页不同。

	\subsection{研究意义} % (fold)
	\label{sub:研究意义}

	% subsection 研究意义 (end)
	\newpage

	\subsection{国内外研究现状} % (fold)
	\label{sub:国内外研究现状}

	% subsection 国内外研究现状 (end)
% section 引言 (end)

\cleardoublepage
\section{\FakeBold{系统设计方案}} % (fold)
\label{sec:系统设计方案}
	\subsection{总体设计方案} % (fold)
	\label{sub:总体设计方案}
	
	% subsection 总体设计方案 (end)
	\newpage

	\subsection{关键问题解决} % (fold)
	\label{sub:关键问题解决}
	
	% subsection 关键问题解决 (end)
	\newpage

	\subsection{设计结果呈现} % (fold)
	\label{sub:设计结果呈现}
	
	% subsection 设计结果呈现 (end)
% section 系统设计方案 (end)

\cleardoublepage
\section{\FakeBold{系统测试、验证与结果分析}} % (fold)
\label{sec:系统测试、验证与结果分析}
	(本部分要求详细记载和描述。)
	\subsection{测试、验证方法} % (fold)
	\label{sub:测试、验证方法}
		(具体方法描述,如参照的标准、测试手段或仪器、方法步骤等)。
	% subsection 测试、验证方法 (end)
	\newpage

	\subsection{测试、验证结果} % (fold)
	\label{sub:测试、验证结果}
		(记录测试数据/结果等。)
	% subsection 测试、验证结果 (end)
	\newpage

	\subsection{测试、验证结果的分析} % (fold)
	\label{sub:测试_验证结果的分析}
	
	% subsection 测试_验证结果的分析 (end)
% section 系统测试、验证与结果分析 (end)

\cleardoublepage
\section{\FakeBold{结论}} % (fold)
\label{sec:结论}
	(对完成工作的总结,可以简述系统设计的结果、预期目标实现情况等;可以列出本文所实现研究成果的一些特点和优点,以及总结完成过程中遇到的一些问题的解决办法,总结本文尚待完善之处,展望对今后改进和完善方向的思考等)
% section 结论 (end)

\cleardoublepage
\section*{\FakeBold{参考文献}} % (fold)
\label{sec:参考文献}
	\addcontentsline{toc}{section}{\protect\numberline{}\FakeBold{参考文献}}%
	“参考文献”四字采用小三号宋体加粗,参考文献采用5号宋体,行间距为1.25倍,文献排列顺序按照在论文正文中出现的先后循序进行,并在正文中将参考文献引用的位置标注出来,两者编号应该一致。

	参考文献的著录规则请参照中华人民共和国国家标准GB/T 7714-2005。个人著者采用姓在前名在后的著录形式。欧美著者的名可以用缩写字母,缩写名后省略缩写点。著作方式相同的责任者不超过3个时,全部照录。超过3个时,只著录前3个责任者,其后加“, 等.”或者“, et al.”等与之相应的词。以下为参考文献格式例子,具体依据个人参考文献。

	\begingroup
	    \setlength{\bibsep}{0pt}
	    \setstretch{1.25}
	    \bibliographyfin{bib/final_report}%
	\endgroup
% section 参考文献 (end)

\cleardoublepage
\section*{\FakeBold{附件}} % (fold)
\label{sec:附件}
	\addcontentsline{toc}{section}{\protect\numberline{}\FakeBold{附件}}%
% section 附件 (end)
