%%%%% --------------------------------------------------------------------------------
%%
%%%%***************************** Literaure Review ***********************************
%%
%%% ++++++++++++++++++++++++++++++++++++++++++++++++++++++++++++++++++++++++++++++++++
\part{文献综述} % (fold)
\label{prt:文献综述_}
	\setcounter{section}{0}
	\section{背景介绍} % (fold)
	\label{sec:背景介绍}
		括号中斜体部分文字为说明文字。

		文献综述是研究者在阅读过某一主题的文献后,经过理解、整理、融会贯通,综合分析和评价而组成的一种不同于研究论文的文体。综述的目的是反映某一研究领域的新水平、新动态、新技术和新发现。一般需要对其从历史到现状、存在问题以及发展趋势等,进行全面的介绍和评论,为开题奠定良好的基础。

		文献综述的撰写遵照“文献综述与开题报告提纲”文件中给出的标题,内容可参考标题后括号内的文字,下级标题可根据综述的内容自行定义。

		文献综述的背景介绍可以包括与课题相关的研究背景,可以介绍有关的概念及定义以及综述的范围、扼要说明有关主题的现状,使读者对全文要叙述的问题有一个初步的轮廓。

		正文文字一级标题用小3号宋体加粗,正文采用小四号宋体,英文及数字采用Times New Roman,行间距为1.5倍行间距。字数要求3000字以上。
	% section 背景介绍 (end)
	\newpage
	\section{国内外研究现状} % (fold)
	\label{sec:国内外研究现状}
		国内外研究现状是文献综述的主体部分,可以包括历史发展、现状分析和研究展望等。在这部分,为了清晰地说明问题,可分为若干个子标题进行阐述。研究现状应将查阅的文献资料进行归纳、整理以及必要的分析,而不是简单的材料的堆砌。写法可以按年代顺序综述,也可按不同的问题进行综述,还可按不同的观点进行比较综述。无论采用方式,都需要做到全面系统、客观公正、分析透彻、层次分明、条理清楚、语言简练、详略得当。以下两个二级标题仅为说明格式要求。

		\subsection{研究方向及进展} % (fold)
		\label{sub:研究方向及进展}
		
		% subsection 研究方向及进展 (end)
		\subsection{存在问题} % (fold)
		\label{sub:存在问题}
		
		% subsection 存在问题 (end)
	% section 国内外研究现状 (end)
	\newpage
	\section*{参考文献} % (fold)
	\label{sec:参考文献1}
	\addcontentsline{toc}{section}{\protect\numberline{}参考文献}%
	
	% section* 参考文献 (end)
% part 文献综述_ (end)

%%% ++++++++++++++++++++++++++++++++++++++++++++++++++++++++++++++++++++++++++++++++++
